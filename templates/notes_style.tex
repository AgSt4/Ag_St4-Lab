% --- PAQUETES BÁSICOS ---
\usepackage{fancyhdr, ragged2e, titlesec, graphicx, enumitem, multicol, pgfplots, subcaption, cancel, listings, float, amsmath, adjustbox, parskip, amssymb, tikz, multirow, booktabs}
\usepackage[dvipsnames]{xcolor}
\usepackage[most]{tcolorbox}
\usetikzlibrary{automata, positioning, shapes, babel}
\usepackage[spanish]{babel}
\pgfplotsset{compat=1.17}
\tcbuselibrary{breakable}

% --- GEOMETRÍA (Necesaria para que el encabezado no se corte) ---
\usepackage[letterpaper, top=0.85in, bottom=1in, left=0.7in, right=0.7in, headheight=50pt, headsep=15pt]{geometry}
\justifying

% --- 1. FUENTES BLINDADAS (Anti-Error en GitHub) ---
\usepackage{iftex}
\iftex
  \usepackage{fontspec}
  \usepackage{unicode-math}
  % Intenta cargar Lato si existe en el sistema (Tu PC)
  \IfFontExistsTF{Lato}{
    \setmainfont{Lato}[
      UprightFont = *-Regular,
      BoldFont = *-Bold,
      ItalicFont = *-Italic,
      BoldItalicFont = *-BoldItalic
    ]
  }{
    % Si no está (GitHub), usa fuentes estándar
    \usepackage[T1]{fontenc}
    \usepackage{lmodern} 
  }
  % Intenta cargar STIX Two Math
  \IfFontExistsTF{STIX Two Math}{
    \setmathfont{STIX Two Math}
  }{}
\else
  % Fallback para compiladores antiguos
  \usepackage[T1]{fontenc}
  \usepackage{lmodern}
\fi

% --- 2. LOGO INTELIGENTE (Graphicspath) ---
% Busca la imagen en la carpeta templates sin importar dónde estés
\graphicspath{{templates/}{../templates/}{../../templates/}{../../../templates/}{../../../../templates/}}

% --- VARIABLES ---
\providecommand{\myCourse}{Curso por Definir}
\providecommand{\mySemester}{Semestre por Definir}
\providecommand{\myEmail}{email@uc.cl}

% --- ENCABEZADO ---
\pagestyle{fancy}
\fancyhead[C]{%
    \begin{minipage}{0.9\textwidth}
        \RaggedRight
        \textit{Pontificia Universidad Católica de Chile} \\
        \textit{Facultad de Ciencias Económicas y Administrativas}
    \end{minipage}}
\fancyhead[L]{%
    \begin{minipage}{0.1\textwidth}
        % OJO: Revisa si tu archivo se llama "Logo UC.png" o "Logo_UC.png"
        \includegraphics[width=0.75cm]{Logo UC.png}
    \end{minipage}}
\fancyhead[R]{%
    \begin{minipage}{0.5\textwidth}
        \raggedleft
        \textit{\myCourse} \\
        \textit{\mySemester}
    \end{minipage}}
\renewcommand{\headrulewidth}{0.5pt}
\renewcommand{\headruleskip}{1.5pt}

% --- CAJAS Y ENTORNOS ---
\newtcolorbox[auto counter, number within=subsection]{respuesta}{%
    breakable, colbacktitle=black!60, colframe=black!60, colback=black!5,
    coltitle=white, fonttitle=\itshape,
    title={Respuesta a la Pregunta \thesubsection},
    arc=5pt, sharp corners = downhill,
    before upper=\setlength{\parskip}{5pt}
}

\newtcbox{\mybox}[1][red]{on line, arc=0pt, outer arc=0pt, colback=#1!10!white, colframe=#1!50!black, boxsep=0pt, left=1pt, right=1pt, top=2pt, bottom=2pt, boxrule=0pt, bottomrule=1pt, toprule=1pt}

% --- CONFIGURACIÓN DE CÓDIGO (Listings) ---
\definecolor{codeblue}{RGB}{0,0,255}
\definecolor{codegreen}{RGB}{0,128,0}
\definecolor{codecrimson}{RGB}{220,20,60}

\lstset{%
    frame=tb, backgroundcolor=\color{white}, aboveskip=3mm, belowskip=3mm,
    showstringspaces=false, columns=flexible, basicstyle={\ttfamily},
    numbers=left, numberstyle=\color{gray}, numbersep=5pt,
    keywordstyle=\color{codeblue}, commentstyle=\color{codegreen},
    stringstyle=\color{codecrimson}, breaklines=true, breakatwhitespace=true,
    tabsize=4, framexleftmargin = -5pt, rulesepcolor =\color{codeblue},
    frame=shadowbox, framesep=4pt, framerule=0.2pt
}

% --- FORMATOS DE TÍTULOS ---
\titleformat{\section}{\normalfont\Large\bfseries\color{RoyalBlue}}{\thesection}{1em}{}
\titleformat{\subsection}{\normalfont\normalsize\bfseries\color{RoyalPurple}}{\thesubsection}{1em}{}
\titleformat{\subsubsection}{\normalfont\normalsize\itshape\color{Periwinkle}}{\thesubsubsection}{1em}{}

% --- TÍTULO PERSONALIZADO ---
\makeatletter
\renewcommand{\maketitle}{%
  \begin{center}
    {\LARGE \textit{\textbf{Notas:} \@title}}\\[3mm]
    \rule{0.8\linewidth}{0.4pt}\\[3mm]
    \begin{tabular}{l l l}
        \textbf{Autor:}& \@author &\href{mailto:\myEmail}{\myEmail} \\[2mm]
        \textbf{Fecha:}& \@date \\[2mm]
    \end{tabular} \\[1mm]
    \rule{0.8\linewidth}{0.4pt}
  \end{center}
  \vspace{5mm}
}
\makeatother